\documentclass[12pt]{article}

\usepackage{amsmath,amsthm,amssymb}
\usepackage{mathtext}
\usepackage[T1,T2A]{fontenc}
\usepackage[utf8x]{inputenc}
\usepackage[english,russian]{babel}
\usepackage{graphics}
\usepackage{graphicx}
\graphicspath{}
\DeclareGraphicsExtensions{.pdf,.png,.jpg}

\newtheorem{theorem}{Лемма}
\begin{document}

\begin{center}
	\hfill \break
	\footnotesize{Санкт-Петербургский политехнический университет}\\ 
	\footnotesize{Высшая школа прикладной математики и вычислительной физики, ФизМех}\\
	\hfill \break
    \hfill \break
	\hfill \break
    \hfill \break
	\hfill\break
	\hfill \break
	\hfill \break
	\hfill \break
	\large{Отчет по лабораторной работе № 3}\\
	\hfill \break
	\large{\textbf{"Решение задачи одномерной минимизации."{}}}\\
	\hfill \break
	\normalsize{дисциплина "Методы оптимизации"{}}\\
	\hfill \break
	\hfill \break
	\hfill \break
	\begin{tabular}{ l l }
		Выполнили студенты гр. 5030102/10401 & \hspace{1cm}Бойцов А.Д.\\
        & \hspace{1cm}Сулейманов В. Д.\\
        & \hspace{1cm}Свитюк А. А.\\
		Преподаватель & \hspace{1cm}Родионова Е. А.
	\end{tabular}
\end{center}

\newpage
\section{Формулировка и формализация задачи}
\subsection{Формулировка задачи}
    \begin{enumerate}
        \item Решить задачу одномерной минимизации, используя метод дихотомии и метод золотого сечения(0.1, 0.01, 0.001 - точности).
        \item Предусмотреть счетчик обращений к вычислению функции и вывод этого числа, необходимого для достижения точности.
        \item Вывести аналитически и привести  формулу числа обращений к вычислению функции для достижения точности.
        \item Показать работу методов на отладочном примере.
        
    \end{enumerate} 
\subsection{Постановка задачи одномерной минимизации}
    Мы будем рассматривать методы, применимые для унимодальных функций. Функция $f(x)$ называется унимодальной на $[a,b]$, если:\\
    \begin{equation}
        \exists ! x^{*}: f(x^{*})=min_{[a,b]} f(x)
    \end{equation}
    причем слева от $x^{*}$ функция монотонно убывает, а справа – монотонно возрастает.\\
    Пусть $f(x)$ - действительная унимодальная функция одной переменной определенная на $[a,b],  a\in R, b\in R$. $f(x^{*})=min_{[a,b]} f(x)$
    
    Тогда задачу одномерной оптимизации для $f(x)$ можно поставить следующим образом:\\
    Положим $\epsilon \in R : 0< \epsilon < b-a$\\
    Трбуется найти
    $\tilde x^{*} \in [a,b]: |\tilde x^{*} - x^{*}|\leq \epsilon $\\
    
\section{Предварительнй анализ}
Оба метода рассматриваемых в работе опираются на следущую лемму:\\
\begin{theorem}\label{t1}
Пусть $f(x)$ - унимодальная функци определенная на $[a,b]$\\
Пусть $x_1, x_2 \in [a,b], x_1 < x_2$\\
Тогда:\\
\begin{enumerate}
    \item $f(x_1)\geq f(x_2) \Rightarrow x^{*} \notin [a,x_1]$
    \item $f(x_1)\leq f(x_2) \Rightarrow x^{*} \notin [x_2,b]$
\end{enumerate}
\end{theorem}

Отрезок $[a,b]$ называется исходным интервалом неопределенности положения
точки минимума на нем. Может теперь сказать, что, анализируя значения функции в точках $x_1, x_2$, мы
сократили интервал неопределенности.\\

\section{Метод золотого сечения}
Идея этого метода также основана на лемме(\ref{t1}): за счет вычисления значения функции в двух точках интервала неопределенности и на основе их сравнения сокращаем интервал неопределенности.\\Будем называть эти точки $f(\lambda_k)$ и $f(\mu_k)$ – левая и права внутренние точки соответственно.
В данном методе на каждой итерации происходит всего 1 вычисление значения функции за счет того, что в качестве одной из внутренних точек, мы берем одну из предыдущих по следующему принципу:
\begin{enumerate}
    \item $\mu_k-a_k=b_k-\lambda_k $
    \item   $
                \left[ 
                  \begin{gathered} 
                    \lambda_k = \mu_{k+1}, \\ 
                    \mu_k = \lambda_{k+1}; \\ 
                  \end{gathered} 
                \right.
            $
\end{enumerate}
Перейдем теперь к параметризации точек. Будем выбирать левую точку следующим
образом:\\
$$
    \lambda_k = a_k +\alpha(b_k-a_k)
$$
Перейдем теперь к параметризации точек. Будем выбирать левую точку следующим
образом:\\
$$
    \mu_k = a_k +(1-\alpha)(b_k-a_k)
$$
Из вышеприведенных условий $\alpha = \dfrac{3-\sqrt{5}}{2}$.
\subsection{Алгоритм метода}
\begin{enumerate}
    \item Зададим $[a,b], a\in R, b\in R $, $f(x)$ - унимодальная функци определенная на $[a,b]$.
    \item Зададим требуемую точность $\epsilon \in R : 0< \epsilon < b-a$.
    \item Положим $\alpha = \dfrac{3-\sqrt{5}}{2}$.
    \item Положим счетчик $k=1$, $a_1 = a, b_1 = b$ 
    \item Положим $\lambda_1 = a_1 +\alpha(b_1-a_1), \mu_1 = = a_1 +(1-\alpha)(b_1-a_1)$
    \item Вычислим значаение функции цели в $\lambda_1,\mu_1$
    \item Если $f(\lambda_k)< f(\mu_k) \Rightarrow$ положим 
    $[a_{k+1},b_{k+1}]=[a_k, \mu_k], \mu_{k+1}=\lambda_k$ и перейдем к шагу 8.\\
    Иначе положим $[a_{k+1},b_{k+1}]=[\lambda_k, b_k], \lambda_{k+1}=\mu_k$ и перейдем к шагу 9.
    \item Увеличим значение счетчика $k+=1$. \\
    Если $b_k-a_k<\epsilon$, заврешиим работу алгоритма.\\
    Иначе положим $f(\mu_k)=f(\lambda_{k-1})$, вычислим $\lambda_k = a_k +\alpha(b_k-a_k)$, $f(\lambda_k)$ и перейдем к шагу 7.
    \item Увеличим значение счетчика $k+=1$. \\
    Если $b_k-a_k<\epsilon$, заврешиим работу алгоритма.\\
    Иначе положим $f(\lambda_k)=f(\mu_{k-1})$, вычислим $\mu_k = a_k +(1-\alpha)(b_k-a_k)$, $f(\lambda_k)$ и перейдем к шагу 7.
\end{enumerate}



\subsection{Аналитическая оценка числа вызовов функции}
\section{Метод дихотомии }
Идея этого метода также основана на лемме(\ref{t1}).На
исходном интервале неопределенности выбираем две точки:\\
$$
    x_1, x_2: x_1<x_2
$$
Рассматривается два случая сравнения значения функции в этих точках:\\
$$
    f(x_1)<f(x_2)\Rightarrow x^{*}\in [a,x_2]\\
    f(x_1)>f(x_2)\Rightarrow x^{*}\in [x_1, b]\\
$$
Полученный интервал будет следующим интервалом неопределенности, с которым
будет проводиться та же самая операция.\\
Очевидно, что оптимальная стратегия выбора точек состоит в следующем:
$$
    x_1,x_2: \min \{\max \{x_2-a, b-x_1\}\}
$$
Отсюда:
$$
    x_1=x_2=\dfrac{b+a}{2}
$$
Но нам нужны две различные точки. Поэтому отступим от найденной точки влево и вправо на какую-то величину $\sigma$:
$$
    x_1=\dfrac{b+a}{2}-\sigma, x_2=\dfrac{b+a}{2}+\sigma
$$
Остается выяснить, каким должно быть $\sigma$. Практический опыт говорит о том, что:
$$
    \sigma	\sim 10^{-3}(b-a)
$$
\subsection{Алгоритм метода}
\begin{enumerate}
    \item Зададим $[a,b], a\in R, b\in R $, $f(x)$ - унимодальная функци определенная на $[a,b]$.
    \item Зададим требуемую точность $\epsilon \in R : 0< \epsilon < b-a$.
    \item Положим $k=1$.
    \item Положим $a_1=a, b_1=b$.
    \item Положим $\sigma = 10^{-3}*(b_k-a_k)$.
    \item Положим $x_{1,k} = \dfrac{b_k+a_k}{2}-\sigma, x_{2,k} = \dfrac{b_k+a_k}{2}+\sigma$.
    \item Вычислим $f(x_{1,k}), f(x_{2,k})$.
    \item Если $f(x_{1,k})<f(x_{2,k})$, положим $[a_{k+1}, b_{k+1}]=[a_k, x_{2,k}]$.\\
    Иначе положим $[a_{k+1}, b_{k+1}]=[x_{1,k}, b_k]$.
    \item Увеличим значение счетчика $k+=1$. \\
    \item Если $b_k-a_k<\epsilon$, заврешиим работу алгоритма.\\
    Иначе перейдем к шагу 5.
    
\end{enumerate}
\subsection{Аналитическая оценка числа вызовов функции}
\section{Тестовый пример}


\end{document}